
\documentclass{article}
\usepackage{polski}
\usepackage[utf8]{inputenc}
\begin{document}

\begin{Huge}
\begin{center}
Gra w życie - sprawozdanie
\end{center}
\end{Huge}
\begin{center}
{\Large Wojciech Musiatowicz i Michał Weiss}
\end{center}

\section{Wprowadzenie}
\subsection*{Ogólne reguły gry}


Gra toczy się na  planszy podzielonej na kwadratowe komórki. Każda komórka ma
ośmiu „sąsiadów” czyli komórki przylegające do niej bokami i rogami. Każda komórka może
znajdować się w jednym z dwóch stanów: może być albo „żywa” (włączona), albo „martwa”
(wyłączona). Stany komórek zmieniają się w pewnych jednostkach czasu. W kolejnych generacjach wszystkie komórki zmieniają swój stan dokładnie w tym samym
momencie. Stan danej komórki zależy od liczby jej żywych sąsiadów.

Martwa komórka, która ma dokładnie 3 żywych sąsiadów, staje się żywa w następnej jednostce
czasu (rodzi się)

Żywa komórka z 2 albo 3 żywymi sąsiadami pozostaje nadal żywa; przy innej liczbie sąsiadów
umiera (z „samotności” albo „zatłoczenia”).

Każda komórka ma dokładnie 8 sąsiadów(lewo,prawo,góra,dół i narożniki). Przypadki szczególne tj. pola na krawędziach planszy zostały rozwiązane zgodnie z poniższymi zasadami:

Pola z górnej krawędzi sąsiadują z polami z dolnej krawędzi (i odwrotnie).Natomiast lewa krawędź sąsiaduje z prawą. Komórki na rogach dodatkowo sąsiadują ze sobą nawzajem na następujących zasadach: 
Pole[0][0] sąsiaduje z pole [h][0]
Pole[0][w] sąsiaduje z polem [h][w]
(dla planszy o wymiarach (h+1) x (w+1))
\section{Struktura programu}
\subsection*{Funkcjonalność}

Program został napisany w języku C. Symuluje grę w życie Johna Connowaya.  Został przewidziany do używania w systemach Windows i Unix. Może być obsługiwany zarówno z linii komend, jak i graficznego interfejsu użytkownika.

Program w zależności od podanego układu komórek i wielkości planszy generuje pliki png zawierające konkretną generację lub podaną liczbę pierwszych generacji. Na utworzonych obrazkach biały kolor reprezentuje martwe komórki, natomiast kolor żywych komórek może ustalić użytkownik.
\subsection*{Działanie programu}
Program pobiera parametry z pliku do odpowiednich struktur. Następnie z kolejnego pliku czytana jest zerowa generacja. Na jej podstawie moduł symulujący tworzy kolejne generacje. W zależności od konfiguracji, przekazywana jest informacja do modułu, który stworzy i zapisze plik
png.
\subsection*{Pliki, ich funkcjonalność oraz struktura}
\subsubsection*{*.cfg}
Plik konfiguracyjny, który można modyfikować.Nazwę pliku, z którego ładowana będzie pierwsza plansza można opcjonalnie podać jako argument linii wiersza poleceń. Domyślna nazwa pliku to "gen.cfg".
Plik zawiera układ komórek, na podstawie której symulowane są kolejne generacje. Komórki reprezentuje macierz znaków plus i minus, gdzie "-" uchodzi za martwą komórkę, a "+" żywą.
\subsubsection*{config.cfg}
Plik konfiguracyjny, który można modyfikować. Zawiera następujące opcje:\\
\emph{numberofgen }- liczba generacji, które chcemy symulować;\\
\emph{print} - numer generacji, którego graficzną reprezentację chcemy zobaczyć. Dla wartości "0" wszystkie generacje są eksportowane do plików png;\\
\emph{red,green,blue} - przyjmują wartości  od 0 do 255, które ustawiają kolor żywych komórek na obrazkach.
\subsubsection*{main.c}
Główny moduł sterujący. 
\subsubsection*{read.h}
Plik nagłówkowy zawierający strukturę:\\
typedef struct \{\par
int height; // wysokość planszy\par
	int width; // szerokość planszy\par
	char **grid; // macierz przechowująca żywe i martwe komórki\par
	int numberofgen;\par
	int print;\par
	int red;\par
	int green;\par
	int blue;\\
\}option;\\
oraz prototypy funkcji \\
int read\_cfg(FILE* config, option *cfg);\\
int read\_grid(FILE* grid, option *cfg);\\
\subsubsection*{read.c}
Moduł odpowiedzialny za czytanie informacji z plików config.cfg i gen.cfg.
\subsubsection*{gen.h}
Plik nagłówkowy do modułu symulującego kolejne generacje.
\subsubsection*{gen.c}
Moduł symulujący generacje. Zawiera funkcje:\\
\emph{int fill (option *cfg) }- odpowiedzialna za poprawne ustawienie komórek na krawędziach planszy, by funkcja generate() mogła sprawdzić ich sąsiedztwo. Funkcja wypełnia zewnętrzną powłokę macierzy hxw, rozszerzając ją tym samym do macierzy (h+2)x(w+2). Dodatkowe pola planszy gry symbolizują sąsiadów pól znajdujących się na krawędziach zasadniczej planszy (rozmiaru hxw). \\
\emph{int  generate(option *cfg)} - odpowiedzialna za sprawdzenie sąsiedztwa każdej komórki i zapisanie tablicy dwuwymiarowej temp[][]. Następnie dane z tablicy nadpisywane są do grid[][].
\subsubsection*{imagen.h}
Plik nagłówkowy do modułu programującego obrazki.
\subsubsection*{imagen.c}
Moduł programujący obrazki png. Zawiera funkcje:\\
int makeimage(option *cfg,int nrgen) - gotowa tablica jest wykorzystana do stworzenia obrazku. Przekazuje do modułu lodepng.c informacje dotyczące obrazka wraz z kolorami oraz jego nazwę.
\subsubsection*{lodepng.h oraz lodepng.c}
Moduł generujący obrazki png. Jest to zewnętrzna biblioteka udostępniona do dowolnego użytku na stronie\emph{ http://lodev.org/lodepng/}
\subsubsection*{Makefile}
Plik kompilujący program.
\subsubsection*{clean.sh, install.sh}
Skrypty umożliwiające korzystanie z programu bez linii komend w środowiskach typu Unix. Clean.sh usuwa zawartośc folderu result, a install.sh kompiluje pliki.
\subsubsection*{clean.bat, install.bat}
Pliki wsadowe umożliwiające korzystanie z programu bez linii komend w systemie operacyjnym Windows.
\pagenumbering{gobble}

  
\end{document}